\begin{frame}[fragile]
  \frametitle{Acquisizione dati}
  Acquisiremo il/i multimetro/i e l'alimentatore in python
  \begin{itemize}
  \item l'acquisizione data si svolger\`a in ambiente Windows
    (aprire il command line di Windows) con python3; createvi ad esempio una cartella
    \begin{verbatim}
    C:/Users/stud02/LabMCS-EsVI 
\end{verbatim}
    \vskip -0.1cm
    (da questa cartella invierete lo script di acquisizione).
  \item l'editing dello script di acquisizione e analisi dati avverr\`a in ambiente WSL (lo stesso che avete sul vostro PC)  nella cartella di cui sopra, accedendovi come:
    \begin{verbatim}
    /mnt/c/Users/stud02/LabMCS-EsVI
    \end{verbatim}
  \end{itemize}
\end{frame}

\begin{frame}[fragile]
  \frametitle{Lettura del multimetro}
  \begin{center}
     \includegraphics[width=0.5\linewidth]{Tek.jpg}
  \end{center}
  \vskip -0.2cm
  I comandi accettati (tra gli altri sono)
  \begin{itemize}
  \item {\tt meas1?} effettua la misura (ritornando il valore e il tipo di misura (ADC/VDC/OHMS))
  \item {\tt range1?} ritorna l'intervallo di misura (utile per stabilire l'errore)
  \end{itemize}
\end{frame}  

\begin{frame}[fragile]
  \frametitle{Lettura del multimetro (modulo {\tt serial})}
  La seriale viene inizializzata con il metodo {\tt Serial} del modulo {\tt serial} a cui bisogna passare (almeno)
  \begin{itemize}
  \item il nome del device ("COMX" con X intero, lo trovare cercando i device su Windows)
  \item la velocit\`a di comunicazione (in bits/sec)
  \end{itemize}
  I comandi vengono passati in python dal modulo con le chiamate
  \begin{itemize}
    \item {\tt write(..)}
    \item {\tt readline()}
  \end{itemize}
  entrambe vogliono in input bytes-array. \\ Quindi a {\tt write} occorre passare {\tt b'comando'}
  e il valore di ritorno di readline deve essere convertito in string con il metodo {\tt decode}.
  
\end{frame}


\begin{frame}[fragile]
  \frametitle{Modulo per il multimetro: {\tt dmm.py}}
Abbiamo creato un modulo {\tt dmm} per facilitare la lettura del multimetro.
\begin{python}
#!/usr/bin/python
import time
import serial
def dmmread(ser):
    # Lettura del valore
    ser.write(b'meas1?\r\n')
    time.sleep(.2)
    line = ser.readline()
    val  = line.decode('ascii').split()
    fval = float(val[0])
    ser.reset_input_buffer()
    ser.reset_output_buffer()
    # Lettura del range
    ser.write(b'range1?\r\n')
    time.sleep(.2)
    line  = ser.readline()
    ival  = line.decode('ascii').split()
    ser.reset_input_buffer()
    ser.reset_output_buffer()
    ind = int(ival[0])-1

    # Calcolo errore
    
    return fval,error
\end{python}
\end{frame}

  \begin{frame}[fragile]
  \frametitle{Modulo per il multimetro: {\tt dmm.py} (II)}

  %  \begin{lstlisting}
  Calcolo dell'errore (essenzialmente trascrittura del manuale dello strumento).
\begin{python}
    error = 0
    if val[1]=="ADC":
        range = [ 0.000200, 0.002000 , 0.020 ,  0.200,  2    , 10  ]
        rel   = [ 0.03    , 0.02    , 0.04  ,  0.03 ,  0.08 , 0.20 ]
        relr  = [ 0.005   , 0.005    , 0.02  ,  0.008,  0.02 , 0.01 ]
        error = abs(fval*rel[ind]/100) + relr[ind]/100*range[ind]
    elif val[1]=="VDC":
        range = [ 0.200,     2 ,    20,  200 ,  1000]
        rel   = [ 0.015 ,  0.015 , 0.015 , 0.015 ,  0.015]
        relr  = [ 0.004,  0.003, 0.004, 0.003, 0.003]  
        error = abs(fval*rel[ind])/100 + relr[ind]/100*range[ind]
    elif val[1]=="OHMS":
        range = [ 200,     2e3 ,    20e3,  200e3 ]
        rel   = [ 0.03 ,  0.02 , 0.02 , 0.02 ]
        relr  = [ 0.004,  0.003, 0.003, 0.003]
        error = abs(fval*rel[ind])/100 + relr[ind]/100*range[ind] + 0.2
\end{python}
\end{frame}

\begin{frame}[fragile]
  \frametitle{Lettura del multimetro (programma principale)}
  Esempio d'uso:
    \begin{lstlisting}
 #!/usr/bin/python
 import sys, serial
 from dmm import *
 ser     = serial.Serial("COM2", 9600)
 val,err = dmmread(ser)
    \end{lstlisting}
  Da completare aggiungendo loop e output su file.
\end{frame}

\begin{frame}[fragile]
  \frametitle{Pilotare il generatore}
  \footnotesize
  \begin{center}
     \includegraphics[width=0.5\linewidth]{Rohde.jpg}
  \end{center}
  \vskip -0.2cm
  Per configurare l'accesso occorre usare il modulo {\tt ResourceManager} di {\tt pyvisa}.
  Quindi i comandi da dare sono:
  \begin{itemize}
    \item  {\tt "INST: NSEL N"}, seleziona il canale N (ce ne sono 2)
    \item  {\tt "INST: OUTN"},   seleziona il canale N in output
    \item  {\tt "OUTP ON"},      abilita l'output del generatore
    \item  {\tt "APPLY V,maxA"}, setta il canale precedentemente selezionato alla tensione V con limite massimo di corrente pari maxA
  \end{itemize}
  L'unico comando da usare in python \`e {\tt write} (poich\'e il generatore \`e molto stabile e preciso e quindi assumiamo il valore settato con errore trascurabile).\\
  Tuttavia, siccome \`e pratico comporre la stringa con il valore di tensione che vogliamo, \`e utile sapere che per inserire un valore numerico in una stringa si usa il prefisso f (float):
\begin{verbatim}
   f'Comando {valore}'
\end{verbatim}
  la stringa che si forma \`e composta dalla stringa alfanumerica pi\`u il valore di val, inserito dove ci sono le parentesi graffe.
\end{frame}
  

\begin{frame}[fragile]
  \frametitle{Modulo per il generatore: {\tt ps.py}}
\begin{python}
#!/usr/bin/python

from   pyvisa import ResourceManager

def init:
    rm = ResourceManager()
    instr = rm.open_resource("USB0::0x0AAD::0x0135::035375056::INSTR")
    return instr

def sel(instr,ch):
    instsel = f"INST: NSEL {ch}"
    instout = f"INST OUT{ch}"
    instr.write(instsel)
    instr.write(instout)
    instr.write("OUTP ON")

\end{python}
\end{frame}

\begin{frame}[fragile]
  \frametitle{Uso del modulo {\tt ps.py}}
\begin{python}
#!/usr/bin/python

from ps import init,sel

instr = ps.init()
ps.sel(instr,1)
valV = 12
cmd = f"APPLY {valV},0.1"
instr.write(cmd)

\end{python}
\end{frame}

\begin{frame}[fragile]
  \frametitle{Misura del tempo}
  La misura del tempo pu\`o rendersi necessaria (ad esempio nell'esperienza dei led con RC).\\
  Il modulo time contiene un metodo {\tt time} che serve allo scopo:
  \begin{itemize}
  \item {\tt time()} ritorna il tempo (in secondi) da un'epoca fissata.\\
    Su Linux/Unix/Windows l'epoca \`e il 1 Gennaio, 1970, 00:00:00.\\
    Per praticit\`a \`e utile calcolare i tempi come differenze di tempo rispetto ad un tempi iniziale.
  \end{itemize}
  
\end{frame}
\end{document}
